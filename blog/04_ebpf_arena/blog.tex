\documentclass{article}
\usepackage{graphicx}
\usepackage{listings}
\usepackage{hyperref}
\usepackage{xcolor}

\title{eBPF Arena: A Tutorial}
\author{Farbod Shahinfar}
\begin{document}
\maketitle

% Inline C code
\newcommand{\myunderscorehavingstring}[1]{{\replunderscores{#1}}}
\makeatletter
% \expandafter for the case that the filename is given in a command
\newcommand{\replunderscores}[1]{\expandafter\@repl@underscores#1_\relax}
\def\@repl@underscores#1_#2\relax{%
    \ifx \relax #2\relax
        % #2 is empty => finish
        #1%
    \else
        % #2 is not empty => underscore was contained, needs to be replaced
        #1%
        \textunderscore
        % continue replacing
        % #2 ends with an extra underscore so I don't need to add another one
        \@repl@underscores#2\relax
    \fi
}
\makeatother
\newcommand*{\C}[1]{\texttt{\myunderscorehavingstring{#1}}}
\newcommand*{\bpfmap}[1]{\C{BPF_MAP_TYPE_\uppercase{#1}}}

\makeatletter
\renewcommand{\href}[2]{\bgroup\let~\H@tilde%
  \Link[#1 target="_blank"]{}{}%
  {#2}\egroup\EndLink}%

\makeatother
\EndPreamble

\definecolor{mygreen}{rgb}{0,0.6,0}
\definecolor{mygray}{rgb}{0.5,0.5,0.5}
\definecolor{mymauve}{rgb}{0.58,0,0.82}
\lstset {
    backgroundcolor=\color{white},  % choose the background color; you must add \usepackage{color} or \usepackage{xcolor}; should come as last argument
    basicstyle=\footnotesize\ttfamily,        % the size of the fonts that are used for the code
    breakatwhitespace=false,         % sets if automatic breaks should only happen at whitespace
    breaklines=false,                 % sets automatic line breaking
    captionpos=b,                    % sets the caption-position to bottom
    commentstyle=\color{mygreen},    % comment style
    deletekeywords={},               % if you want to delete keywords from the given language
    escapeinside={\%*}{*)},          % if you want to add LaTeX within your code
    extendedchars=true,              % lets you use non-ASCII characters; for 8-bits encodings only, does not work with UTF-8
    firstnumber=1,                   % start line enumeration with line 1000
    frame=single,                    % adds a frame around the code
    keepspaces=false,                 % keeps spaces in text, useful for keeping indentation of code (possibly needs columns=flexible)
    keywordstyle=\color{blue},       % keyword style
    language=c,                      % the language of the code
    % if you want to add more keywords to the set
    morekeywords={inline, uint8_t, uint16_t, uint32_t, __u8, __u16, __u32, __u64},
    numbers=left,                    % where to put the line-numbers; possible values are (none, left, right)
    numbersep=5pt,                   % how far the line-numbers are from the code
    numberstyle=\tiny\color{black}, % the style that is used for the line-numbers
    rulecolor=\color{red},         % if not set, the frame-color may be changed on line-breaks within not-black text (e.g. comments (green here))
    showspaces=false,                % show spaces everywhere adding particular underscores; it overrides 'showstringspaces'
    showstringspaces=false,          % underline spaces within strings only
    showtabs=false,                  % show tabs within strings adding particular underscores
    stepnumber=1,                    % the step between two line-numbers. If it's 1, each line will be numbered
    stringstyle=\color{mymauve},     % string literal style
    tabsize=1,                       % sets default tabsize to 2 spaces
    % title=\lstname,                  % show the filename of files included with \lstinputlisting; also try caption instead of title
}

\Css{
body{
    width: 40em;
    margin: auto;
    display: flex;
    flex-direction: column;
    adjust-contenct: center;
    justify-content: center;
    font-size: 18pt;
}
}


As part of my research, I spend much time learning about eBPF.
Recently, I am busy exploring \emph{Arena} --- a new eBPF API that enables
programs to allocate memory pages; similar functionallity as \C{mmap} and
\C{munmap}~\cite{arenapatch}.

The eBPF community is doing a great job documenting the system and writing
tutorials. I felt I could contribute to this effort by writing about Arena.

{
    \textcolor{red}{At the moment, this blog post is a \emph{work in
    progress} and I will update it as I learn more about Arena and figure
    things out.}
}

\section{Introduction}

The \emph{Arena} is a new MAP type (\bpfmap{arena}) available to eBPF programs
since kernel version 6.9.
This map is semantically different than previous ones.
Unlike data-struture MAPs (e.g., hash, array, bloom, stack, \dots{}) Arena
provides a direct access to the kernel memory instead of abstracting it away.
It increases the programs expressivitiy and enables them to implement their own
data-structures on-demand (e.g, a specific type of tree or a more optimized
version of hash-map).

Three use-cases have been named for the Arena in its patch introduction
message~\cite{arenapatch}. I summarize them bellow:
\begin{itemize}
    \item User-space can use Arena's memory to implement its own
        data-structures (it is a normal memory region), while the memory is
        visible to eBPF programs which can implement a fast-path for operations
        like packet processing.
    \item Arena can be used as a communication channel between an eBPF and a
        user-space program.
    \item Arena can be used as a \emph{heap} of memory for the eBPF program
\end{itemize}
My understanding from these use-cases is that Arena provides access to raw
pages of memory shareable among eBPF programs and user-sace. Each can decide to
use it as they see fit with their own responsibility. It is unlike other types
of data-structure MAPs that enforce certain constraints such as fixed key and
value sizes while limited the operations to lookups and updates.

In this tutorial, I try to explore Arena's API, and provide an example for each
use-case in the hope that it facilitates the journey of other in figuring out
how to use this new capability. The code snippets shown in this text is shared
on \href{https://github.com/bpf-endeavor/ebpf-arena-tutorial}{this repository}.

\section{Arena APIs}

A program can declare the Arena map as other maps. See Listing~\ref{lst:map_ex}
for an example. The key size and value sizes must be zero. Max entries define
the maximum number of pages for the map, and must be non-zero. As any other
map, the maximum size is limited to 4~GB. At the time of writing this text,
Arena map declaration supports three flags~\cite{arena_source}:
\begin{itemize}
    \item \C{BPF_F_MMAPABLE}: \textbf{Required}. Indicates the map should
        support memory mapping.
    \item \C{BPF_F_SEGV_ON_FAULT}: The user-space program will receive a
        \C{SEGFAULT} signal if the memory is not mapped by the eBPF program
        (e.g., out of range access on a list created by the eBPF program).
    \item \C{BPF_F_NO_USER_CONV}: The memory region is private to the eBPF
        program and user-space program will not access it. This allows the JIT
        to perform optimizations avoiding address space casting (more about it
        later).
\end{itemize}
\textcolor{gray}{\textit{note: The \C{BPF_F_MMAPABLE} must always be present.}}

\begin{minipage}{\linewidth}
    \centering
    \begin{flushleft}
        \begin{lstlisting}[caption={Example of using Arena map}, label={lst:map_ex}]
        struct {
         __uint(type, BPF_MAP_TYPE_ARENA);
         __uint(map_flags, BPF_F_MMAPABLE);
         __uint(max_entries, 2); /* number of pages */
        } arena SEC(".maps");
        \end{lstlisting}
    \end{flushleft}
\end{minipage}

Normal eBPF map helpers such as ``\C{bpf_map_lookup_elem}'' are not
defined for Arena~\cite{arena_source}. Instead the following pair of functions
are available:
\begin{itemize}
    \item \C{bpf_arena_alloc_pages}: Allocates a memory page
    \item \C{bpf_arena_free_pages}: Free the memory page
\end{itemize}

These functions are defined using kfunc~\cite{eunomia_kfunc, ebpf_docs_kfunc}.
As required by kfunc subsystem, you should declare the signature of these
helpers in your eBPF program to ues these functions.
\begin{minipage}{\linewidth}
    \begin{flushleft}
            \begin{lstlisting}
            void __arena* bpf_arena_alloc_pages(void *map, void __arena *addr,
                __u32 page_cnt, int node_id, __u64 flags) __ksym;
            void bpf_arena_free_pages(void *map, void __arena *ptr,
                __u32 page_cnt) __ksym;
            \end{lstlisting}
        \end{flushleft}
\end{minipage}

Calling these functions may put the thread to sleep. For this reason the
functions are marked with \C{KF_SLEEPABLE}~\cite{arena_source}, and the
verifier only allows sleepable eBPF programs to use them. An eBPF program has
sleeping privilege if the \C{BPF_F_SLEEPABLE} flag was set when loading it
(\href{https://github.com/bpf-endeavor/ebpf-arena-tutorial/blob/9b399ae7784ade572aa213f0ac37fd2d84b7c13a/src/mogu_loader.c\#L91}{an example loader}).
Not all eBPF hooks support this flag. A list of sleepable
hooks are provided in \emph{libbpf} documentations~\cite{libbpf_sleepable}.

\textcolor{gray}{\emph{note: The memory pages allocated use following flags: \C{GFP_KERNEL}, \C{__GFP_ZERO}, \C{__GFP_ACCOUNT}.}}


\section{Example}
\dots{}

% \lstinputlisting[caption={a}]{./mogu.bpf.c}

\bibliographystyle{plain}
\bibliography{ref}
\end{document}
