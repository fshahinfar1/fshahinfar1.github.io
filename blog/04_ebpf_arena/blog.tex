\documentclass{article}
\usepackage{graphicx}
\usepackage{hyperref}
\usepackage{xcolor}
\usepackage{minted}

\title{eBPF Arena: A Tutorial}
\author{Farbod Shahinfar}
\begin{document}
\maketitle

% Inline C code
\newcommand{\myunderscorehavingstring}[1]{{\replunderscores{#1}}}
\makeatletter
% \expandafter for the case that the filename is given in a command
\newcommand{\replunderscores}[1]{\expandafter\@repl@underscores#1_\relax}
\def\@repl@underscores#1_#2\relax{%
    \ifx \relax #2\relax
        % #2 is empty => finish
        #1%
    \else
        % #2 is not empty => underscore was contained, needs to be replaced
        #1%
        \textunderscore
        % continue replacing
        % #2 ends with an extra underscore so I don't need to add another one
        \@repl@underscores#2\relax
    \fi
}
\makeatother
\newcommand*{\C}[1]{\texttt{\myunderscorehavingstring{#1}}}
\newcommand*{\bpfmap}[1]{\C{BPF_MAP_TYPE_\uppercase{#1}}}

\makeatletter
\renewcommand{\href}[2]{\bgroup\let~\H@tilde%
  \Link[#1 target="_blank"]{}{}%
  {#2}\egroup\EndLink}%

\makeatother
\EndPreamble

\definecolor{mygreen}{rgb}{0,0.6,0}
\definecolor{mygray}{rgb}{0.5,0.5,0.5}
\definecolor{mymauve}{rgb}{0.58,0,0.82}
\lstset {
    backgroundcolor=\color{white},  % choose the background color; you must add \usepackage{color} or \usepackage{xcolor}; should come as last argument
    basicstyle=\footnotesize\ttfamily,        % the size of the fonts that are used for the code
    breakatwhitespace=false,         % sets if automatic breaks should only happen at whitespace
    breaklines=false,                 % sets automatic line breaking
    captionpos=b,                    % sets the caption-position to bottom
    commentstyle=\color{mygreen},    % comment style
    deletekeywords={},               % if you want to delete keywords from the given language
    escapeinside={\%*}{*)},          % if you want to add LaTeX within your code
    extendedchars=true,              % lets you use non-ASCII characters; for 8-bits encodings only, does not work with UTF-8
    firstnumber=1,                   % start line enumeration with line 1000
    frame=single,                    % adds a frame around the code
    keepspaces=false,                 % keeps spaces in text, useful for keeping indentation of code (possibly needs columns=flexible)
    keywordstyle=\color{blue},       % keyword style
    language=c,                      % the language of the code
    % if you want to add more keywords to the set
    morekeywords={inline, uint8_t, uint16_t, uint32_t, __u8, __u16, __u32, __u64},
    numbers=left,                    % where to put the line-numbers; possible values are (none, left, right)
    numbersep=5pt,                   % how far the line-numbers are from the code
    numberstyle=\tiny\color{black}, % the style that is used for the line-numbers
    rulecolor=\color{red},         % if not set, the frame-color may be changed on line-breaks within not-black text (e.g. comments (green here))
    showspaces=false,                % show spaces everywhere adding particular underscores; it overrides 'showstringspaces'
    showstringspaces=false,          % underline spaces within strings only
    showtabs=false,                  % show tabs within strings adding particular underscores
    stepnumber=1,                    % the step between two line-numbers. If it's 1, each line will be numbered
    stringstyle=\color{mymauve},     % string literal style
    tabsize=1,                       % sets default tabsize to 2 spaces
    % title=\lstname,                  % show the filename of files included with \lstinputlisting; also try caption instead of title
}

\Css{
body{
    width: 40em;
    margin: auto;
    display: flex;
    flex-direction: column;
    adjust-contenct: center;
    justify-content: center;
    font-size: 18pt;
}
}


As part of my research, I spend much time learning about eBPF.
Recently, I am busy exploring \emph{Arena} --- a new eBPF API that enables
programs to allocate memory pages; similar functionality as \C{mmap} and
\C{munmap}~\cite{arenapatch}.

The eBPF community is doing a great job documenting the system and writing
tutorials. I felt I could contribute to this effort by writing about Arena.

{
    \textcolor{red}{At the moment, this blog post is a \emph{work in
    progress} and I will update it as I learn more about Arena and figure
    things out.}
}

\section{Introduction}

The \emph{Arena} is a new MAP type (\bpfmap{arena}) available to eBPF programs
since kernel version 6.9.
This map is semantically different than previous ones.
Unlike data-structure MAPs (e.g., hash, array, bloom, stack, \dots{}) Arena
provides a direct access to the kernel memory instead of abstracting it away.
It increases the programs expressivity and enables them to implement their own
data-structures on-demand (e.g, a specific type of tree or a more optimized
version of hash-map).

Three use-cases have been named for the Arena in the introduction
message~\cite{arenapatch} sent along its patch to the Linux mailing list. I
summarize them bellow:
\begin{itemize}
    \item User-space can use Arena's memory to implement its own
        data-structures (it is a normal memory region), while the memory is
        visible to eBPF programs which can implement a fast-path for operations
        like packet processing.
    \item Arena can be used as a communication channel between an eBPF and a
        user-space program.
    \item Arena can be used as a \emph{heap} of memory for the eBPF program
\end{itemize}
My understanding from these use-cases is that Arena provides access to raw
pages of memory shareable among eBPF programs and user-space. Each can decide
to use it as they see fit with their own responsibility. It is unlike other
types of data-structure MAPs that enforce certain constraints such as fixed key
and value sizes, or limiting the operations to lookup and update.

In this tutorial, I try to explore Arena's API, and provide some examples using
it. I hope this document help others explore Arena and facilitate its
development. The code snippets shown in this text is shared on
\href{https://github.com/bpf-endeavor/ebpf-arena-tutorial}{this repository}.

\section{Arena's API}

\subsection{Declare a MAP}
A program can declare the Arena MAP similar to other MAPs. See Listing~\ref{lst:map_ex}
for an example. Here are some notes to keep in mind. The key size and value
sizes must be zero. Max entries define the maximum number of pages for the map,
and must be non-zero. As any other map, the maximum size is limited to 4~GB. At
the time of writing this text, Arena map declaration supports three
flags~\cite{arena_source}:
\begin{itemize}
    \item \C{BPF_F_MMAPABLE}: \textbf{Required}. Indicates the map should
        support memory mapping.
    \item \C{BPF_F_SEGV_ON_FAULT}: The user-space program will receive a
        \C{SEGFAULT} signal if the memory is not mapped by the eBPF program
        (e.g., out of range access on a list created by the eBPF program).
    \item \C{BPF_F_NO_USER_CONV}: The memory region is private to the eBPF
        program and user-space program will not access it. This allows the JIT
        to perform optimizations avoiding address space casting (more about it
        later).
\end{itemize}
\textcolor{gray}{\textit{note: The \C{BPF_F_MMAPABLE} must always be present.}}

\centering
\begin{listing}
\begin{minted}{c}
struct {
    __uint(type, BPF_MAP_TYPE_ARENA);
    __uint(map_flags, BPF_F_MMAPABLE);
    __uint(max_entries, 2); /* number of pages */
} arena SEC(".maps");
\end{minted}
    \caption{Example of using Arena map.}
    \label{lst:map_ex}
\end{listing}

\subsection{Helper functions}
Normal eBPF MAP helpers such as ``\C{bpf_map_lookup_elem}'' are not
defined for Arena~\cite{arena_source}. Instead the following pair of functions
are available:
\begin{itemize}
    \item \C{bpf_arena_alloc_pages}: Allocates a memory page
    \item \C{bpf_arena_free_pages}: Frees the memory page
\end{itemize}

These functions are defined using kfunc~\cite{eunomia_kfunc, ebpf_docs_kfunc}.
As required by kfunc subsystem, you should declare the signature of these
helpers in your eBPF program to ues these functions.
\begin{listing}
\begin{minted}{c}
void __arena* bpf_arena_alloc_pages(void *map, void __arena *addr,
    __u32 page_cnt, int node_id, __u64 flags) __ksym;
void bpf_arena_free_pages(void *map, void __arena *ptr,
    __u32 page_cnt) __ksym;
\end{minted}
\caption{Functions operating on Arena MAP.}
\label{lst:arena_kfuncs}
\end{listing}

Calling these functions may put the thread to sleep. For this reason the
functions are marked with \C{KF_SLEEPABLE}~\cite{arena_source}, and the
verifier only allows sleepable eBPF programs to use them. An eBPF program has
sleeping privilege if the \C{BPF_F_SLEEPABLE} flag was set when loading it
(see Listing~\ref{lst:loader} for an example).
Not all eBPF hooks support this flag. A list of sleepable
hooks are provided in \emph{libbpf} documentations~\cite{libbpf_sleepable}.

\textcolor{gray}{\emph{note: The memory pages allocated use following flags:
\C{GFP_KERNEL}, \C{__GFP_ZERO}, \C{__GFP_ACCOUNT}.}}

\subsection{Managing two address spaces}

\emph{Clang} compiler, the verifier, the JIT, and the runtime all work together
to make sure the eBPF program accesses the correct memory address. When using
Arena there is a translation between the user address space and kernel address
space (unless the map is declared with ``\C{BPF_F_NO_USER_CONV}'' flag
present). The eBPF program must mark the pointers with
``\C{__attribute__((address_space(1)))}'' to let the Clang know about it and
cause the generation of ``\C{bpf_arena_cast_user/kern}'' instrucitons. The
``\C{__arena}'' used in declaration of variables and parameters bears this
purpose.

\textcolor{gray}{\emph{note: if this explanation is vague, it is because I need
to learn more about it.}}


\section{Examples}

I try to demonstrate how these APIs can be used to implement the use-cases
named above.

\subsection{Memory shared among eBPF and user-space}

As a first step, let us demonstrate how to share a counter between an eBPF
and an user-space program. Let us define a eBPF program of type
\C{BPF_PROT_TYPE_SYSCALL}~\cite{ebpf_docs_prog_syscall} which does not need to
be attached to a hook and instead is invoked using the \C{BPF_PROG_TEST_RUN}
system call~\cite{ebpf_docs_bpf_prog_run}.

Listing~\ref{lst:mogu} demonstrates an eBPF program using Arena. In this
program, a global variable (``\C{flag_initialized}'') tracks if it is the first
time the program is invoked or not. At the first invocation a page is allocated
and its address is kept in ``\C{mem}'' global variable. The allocated memory
will hold a counter keeping the number of invocations.

\textcolor{gray}{\textit{note: The global variable support is available since
kernel version 5.2~\cite{glb_var_post}.}}

\begin{listing}
\begin{minted}{c}
/* Declaration of Arena MAP as in Listing 1 ... */
static bool flag_initialized = false;
__arena void *mem = NULL;
SEC("syscall")
int mogu_main(void *_)
{
    bpf_printk("mogu: hello\n");
    if (!flag_initialized) {
        mem = bpf_arena_alloc_pages(&arena, NULL, 1, NUMA_NO_NODE, 0);
        if (mem == NULL) {
            bpf_printk("Failed to allocate memory");
            return 1;
        }
        flag_initialized = true;
    }
    if (mem == NULL) {
        /* this branch must never happen! */
        return 1;
    }
    __arena entry_t *e = mem;
    e->counter += 1;
    bpf_printk("counter: %lld\n", e->counter);
    return 0;
}
\end{minted}
\caption{An eBPF program using Arena.}
\label{lst:mogu}
\end{listing}

Next, in Listing~\ref{lst:loader}, we see how user-space program loads the eBPF
program and Listing~\ref{lst:loader_read} shows it accessing the counter on the
Arena memory page. The user-space program relies on skeleton objects from
libbpf~\cite{libbpf_skeleton}.

\begin{listing}
\begin{minted}{c}
/* Some global vars */
static volatile int running = 0;
static int ebpf_prog_fd = -1;
static struct mogu *skel = NULL;

int main(int argc, char *argv[])
{
    skel = mogu__open();
    if (!skel) {
        fprintf(stderr, "Failed to open the skeleton\n");
        return EXIT_FAILURE;
    }
    /* Set sleepable flag */
    bpf_program__set_flags(skel->progs.mogu_main, BPF_F_SLEEPABLE);
    if (mogu__load(skel)) {
        fprintf(stderr, "Failed to load eBPF program\n");
        return EXIT_FAILURE;
    }

    ebpf_prog_fd = bpf_program__fd(skel->progs.mogu_main);
    /* It will invoke the eBPF program for the first time */
    handle_invoke_signal(0);

    /* Keep running and handle signals */
    running = 1;
    signal(SIGINT, handle_signal);
    signal(SIGHUP, handle_signal);
    signal(SIGUSR1, handle_invoke_signal);
    printf("Hit Ctrl+C to terminate ...\n");
    printf("Invoke eBPF program:\n");
    printf("\tMogu: pkill -SIGUSR1 mogu_loader\n");

    while (running) { pause(); }

    mogu__detach(skel);
    mogu__destroy(skel);
    printf("Done!\n");
    return 0;
}
\end{minted}
\caption{User space program loading the program}
\label{lst:loader}
\end{listing}

\begin{listing}
\begin{minted}{c}
entry_t *e = skel->bss->mem;
if (e == NULL) {
    printf("NOTE: the initialization was not successful!\n");
    return;
}
printf("user: counter=%lld\n", e->counter);
\end{minted}
\caption{User-space accessing the memory page allocated from Arena}
\label{lst:loader_read}
\end{listing}


\subsection{Using Arena in XDP (or other hooks that can not sleep)}

As was mentioned, Arena helper functions may put the calling thread to sleep.
This is not acceptable for program attached to hooks such as \emph{XDP} in
which the program is running in a \C{NAPI}. But, this limitation does not mean
that XDP programs can not access the arena memory. They just can not allocate or
free pages.

For the next step, Let us share the memory page allocated by the eBPF program
from the previous example with another eBPF program which does not use the Arena
helpers.

Listing~\ref{lst:aloe} shows the second eBPF program. It does not use the
\C{bpf_arena_alloc_pages} helper for allocating the memory. Instead, it is
using the memory page allocated before and access the counter value, both
reading and writing to it.

Here are some details that I like to bring to your attention:
\begin{enumerate}
    \item Notice that both eBPF programs declare an Arena. The loader will
        make sure both are using the same MAP using the libbpf's
        \C{bpf_map__reuse_fd} helper.
    \item The second program also expects the loader to pass the address of
        allocated page through its global variable called ``mem''. This address
        could have been shared among the two program in other ways, e.g.
        through a shared array MAP\@.
    \item The second program is using a helper (\C{my_kfunc_reg_arena}) that is
        not part of the Linux kernel. This function does not perform any
        operation. Its sole role is to let the verifier recognize that the
        program is using the Arena. See \S\ref{sec:need_custom} for more detail.
\end{enumerate}

\begin{listing}
\begin{minted}{c}
/* Declaration of Arena MAP as in Listing 1 ... */
__arena void *mem = NULL;

/* Load the kernel module for adding this kfunc */
long my_kfunc_reg_arena(void *p__map) __ksym;

SEC("syscall")
int aloe_main(void *_)
{
    my_kfunc_reg_arena(&arena_map);
    if (mem == NULL) {
        bpf_printk("aloe: not seeing the memory!\n");
        return 1;
    }
    __arena entry_t *e = mem;
    bpf_printk("aloe: counter=%lld\n", e->counter);
    e->counter += 100;
    return 0;
}
\end{minted}
\caption{An eBPF program that uses Arena pages allocated from another program}
\label{lst:aloe}
\end{listing}

To share the Arena from the first eBPF program (Listing~\ref{lst:mogu})
with the second one (Listing~\ref{lst:aloe}), the loader (user-space) program
has to mediate and asign the reference. Listing~\ref{lst:share-map-ref} is an
example of this procedure.

\begin{listing}
\begin{minted}{c}
int arena_fd = bpf_map__fd(skel1->maps.arena);
bpf_map__reuse_fd(skel2->maps.arena_map, arena_fd);
/* Pass the pointer to the  second program */
skel2->bss->mem = skel1->bss->mem;
\end{minted}
\caption{Loader program assigning the Arena from the first program to the second program}
\label{lst:share-map-ref}
\end{listing}

\subsubsection{Why do we need a custom Arena function?}
\label{sec:need_custom}
If we remove the the call to the \C{my_kfunc_reg_arena} in
Listing~\ref{lst:aloe}, although the program would compile, the verifier will
complain with the following message:

\begin{listing}
\begin{minted}{c}
addr_space_cast insn can only be used in a program that has an associated arena
\end{minted}
\caption{Verifier error message when the Arena is not referenced in the program}
\label{lst:verifier_err_msg}
\end{listing}

Looking at the verifiers source code, the error is raised when the program
is not associated with an Arena~\cite{verifier_arena_not_set}. This association
happens when the MAP is referenced (e.g., by using a helper function).
In the second program, we deliberately do not want to use Arena specific
helpers. Other helper functions such as \C{bpf_map_lookup_elem} also cause
other verifier complains (in my test it complained due to attempting a zero
byte read). Defining a new helper that does nothing but still satisfies the
verifier is a smart hack, which also show cases the flexibility that kfunc
support has brought to eBPF.

\textcolor{gray}{\textit{note: The idea of having a dummy helper was suggested
by my labmate \href{https://github.com/marcomole00/}{Marco Mole }.}}


\subsection{User-space managed data-structures}

% \C{bpf_map__initial_value}
\dots{}


\bibliographystyle{plain}
\bibliography{ref}
\end{document}
