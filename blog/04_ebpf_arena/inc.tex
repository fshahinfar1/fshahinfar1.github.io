% Inline C code
\newcommand{\myunderscorehavingstring}[1]{{\replunderscores{#1}}}
\makeatletter
% \expandafter for the case that the filename is given in a command
\newcommand{\replunderscores}[1]{\expandafter\@repl@underscores#1_\relax}
\def\@repl@underscores#1_#2\relax{%
    \ifx \relax #2\relax
        % #2 is empty => finish
        #1%
    \else
        % #2 is not empty => underscore was contained, needs to be replaced
        #1%
        \textunderscore
        % continue replacing
        % #2 ends with an extra underscore so I don't need to add another one
        \@repl@underscores#2\relax
    \fi
}
\makeatother
\newcommand*{\C}[1]{\texttt{\myunderscorehavingstring{#1}}}
\newcommand*{\bpfmap}[1]{\C{BPF_MAP_TYPE_\uppercase{#1}}}

\makeatletter
\renewcommand{\href}[2]{\bgroup\let~\H@tilde%
  \Link[#1 target="_blank"]{}{}%
  {#2}\egroup\EndLink}%

\makeatother
\EndPreamble

\definecolor{mygreen}{rgb}{0,0.6,0}
\definecolor{mygray}{rgb}{0.5,0.5,0.5}
\definecolor{mymauve}{rgb}{0.58,0,0.82}
\lstset {
    backgroundcolor=\color{white},  % choose the background color; you must add \usepackage{color} or \usepackage{xcolor}; should come as last argument
    basicstyle=\footnotesize\ttfamily,        % the size of the fonts that are used for the code
    breakatwhitespace=false,         % sets if automatic breaks should only happen at whitespace
    breaklines=false,                 % sets automatic line breaking
    captionpos=b,                    % sets the caption-position to bottom
    commentstyle=\color{mygreen},    % comment style
    deletekeywords={},               % if you want to delete keywords from the given language
    escapeinside={\%*}{*)},          % if you want to add LaTeX within your code
    extendedchars=true,              % lets you use non-ASCII characters; for 8-bits encodings only, does not work with UTF-8
    firstnumber=1,                   % start line enumeration with line 1000
    frame=single,                    % adds a frame around the code
    keepspaces=false,                 % keeps spaces in text, useful for keeping indentation of code (possibly needs columns=flexible)
    keywordstyle=\color{blue},       % keyword style
    language=c,                      % the language of the code
    % if you want to add more keywords to the set
    morekeywords={inline, uint8_t, uint16_t, uint32_t, __u8, __u16, __u32, __u64},
    numbers=left,                    % where to put the line-numbers; possible values are (none, left, right)
    numbersep=5pt,                   % how far the line-numbers are from the code
    numberstyle=\tiny\color{black}, % the style that is used for the line-numbers
    rulecolor=\color{red},         % if not set, the frame-color may be changed on line-breaks within not-black text (e.g. comments (green here))
    showspaces=false,                % show spaces everywhere adding particular underscores; it overrides 'showstringspaces'
    showstringspaces=false,          % underline spaces within strings only
    showtabs=false,                  % show tabs within strings adding particular underscores
    stepnumber=1,                    % the step between two line-numbers. If it's 1, each line will be numbered
    stringstyle=\color{mymauve},     % string literal style
    tabsize=1,                       % sets default tabsize to 2 spaces
    % title=\lstname,                  % show the filename of files included with \lstinputlisting; also try caption instead of title
}

\Css{
body{
    width: 40em;
    margin: auto;
    display: flex;
    flex-direction: column;
    adjust-contenct: center;
    justify-content: center;
    font-size: 18pt;
}
}
